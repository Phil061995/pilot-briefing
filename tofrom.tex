%!TEX root = EDDF_Pilot_Briefing.tex
\subsection[DEL]{Startup and IFR Clearance [Delivery]}
For your IFR-clearance, first of all listen to Frankfurt ATIS to
get weather information and the active runways, as well as
the ATIS letter.\\
Clearance Delivery will issue you an IFR-clearance containing the Standard Instrument Departure (SID) to fly and the transponder code. Below, find a table of SIDs used most often.
\begin{table}[h]
	\begin{tabular}{|l|l|l|l|}
		\hline
		\textbf{Waypoint} & \textbf{SID}  & \textbf{RWY} & \textbf{CLB}\\\hline
		MARUN    & M 		& 25 & FL70			\\
				 & F/G*		& 25 & 5000ft* 		\\
				 & E/D		& 07 & 5000ft   	\\\hline
		OBOKA    & M 		& 25 & FL70			\\
				 & F/G*		& 25 & 5000ft* 		\\
				 & E/D		& 07 & 5000ft   	\\\hline
		TOBAK    & M    	& 25 & FL70	\\
				 & F/G*		& 25 & 5000ft* \\
				 & D		& 07 & 5000ft   	\\\hline
		ULKIG    & L   		& 18 & 4000ft 		\\\hline
		SOBRA    & L    	& 18 & 4000ft 		\\\hline
		ANEKI    & L    	& 18 & 4000ft 		\\\hline
		CINDY    & S    	& 18 & 4000ft 		\\\hline
		SULUS    & S    	& 18 & 4000ft  		\\
				 & D    	& 07 & 4000ft 		\\
		\hline
	\end{tabular}
\end{table}

* Two engine heavy aircraft or if unable.\newpage
Note "startup approved" means that you can expect departure in the next 20 minutes, please \textbf{do not mistake this for pushback clearance}! Start your engines at your own discretion - typically, while pushing back or shortly after
pushback.
If unable to fly SIDs, request a radar vectored departure. In this case, the controller will tell you to fly runway heading and maintain either 5000ft or FL070 for 07/25-departure or 4000ft for 18-departures.\\
For valid routes see \url{http://grd.aero-nav.com/?From=EDDF}

\subsection[GND]{Pushback and Taxi [Apron,Ground]}
\paragraph{General}
Have a chart of the airport ready, listen carefully and readback all taxiways and hold short instructions. In case of doubt, hold your position and ask ATC for help.

\paragraph{Departures}
"Delivery" will instruct you to contact "Frankfurt Apron" to request pushback (or taxi in case you are on a taxi-out position). Attention: There is more than one apron responsible for pushback and taxi, so listen carefully for the correct frequency.
For pushback and taxi, you always need a clearance.
Set Squawk Mode C as soon as you start your pushback/taxi.
Please do not waste any time and start your pushback as soon as possible, thereafter, also as soon as possible, request taxi.

If your gate/stand is located in the east but the given departure runway is 18, it is quite common to send
you via taxiways U, S, S11, R, S25 and S for an intersection departure via S. This is due to congestion on the main Apron.

\paragraph{Arrivals}
Arrivals entering the apron via P (from RWY 07L/25R) can expect "hold short of N11".

\subsection[TWR]{Takeoff and Landing [Tower]}

\paragraph{Departures}
Different holding points may be used to sequence departing traffic, stick to the ground charts!
If you want to do an intersection departure, tell the controller! ATC may send light and medium
aircraft to an intersection directly, such as L6 for runway 25C.
Check that your transponder is set prior to departure.
When airborne, Tower will hand you off to "Langen Radar" (DEP/APP).

\paragraph{Arrivals}
For landing you will get a landing-clearance from TWR.
After you landed vacate the runway as fast as possible.
If you land on RWY 25L/07R vacate to the north (except you want to park in the south or ATC tells you).
Tower will also will give you the first taxi instructions.
Stay on tower frequency until you get a handoff to tower, ground or apron controller.

\subsection[APP]{Approach and Departure [Approach,Director,Departure]}
\paragraph{Departures}
Please make sure not to excede your initial climb unless cleared by ATC.
Otherwise you may cause conflicts with arriving traffic!
Transition altitude is 5000ft.

\paragraph{Arrivals}
Thanks to airspace structure, there are NO speed restrictions for IFR-traffic below FL100 unless
instructed by ATC.
Check the ATIS for the current transition level and report ATIS Letter on initial contact.
Approach will get you on the downwind and hand you over to Director controller.\\
On initial contact with Frankfurt Director (EDDF\_ F/U\_APP) state callsign ONLY (just “Frankfurt Director, <callsign>”).
Do not report “established on ILS”! During busy times, expect to be told to maintain speed 170 IAS or greater until 5 miles final / 5 DME / Final Approach Point (FAP), following traffic will be only a few miles behind!

\subsection[CTR]{Cruise [Radar]}
\paragraph{General}
On initial call, always report current fightlevel (and cleared flightlevel).

\paragraph{Arrivals}
STARs are not used in Frankfurt. Always expect transition or vectors.
If you arrive from the north expect ROLIS/UNOKO/KERAX 07/25 \textbf{N} transition.
If you arrive from the south expect SPESA (formerly PSA) 07/25 \textbf{S} transition.
ATC may instruct "when ready descend to reach flight level ... at ..." make sure you start your descent accordingly.
